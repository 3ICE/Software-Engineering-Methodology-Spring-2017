\section{Risk Management Plan}

\begin{table}[h]
\small
\captionof{table}{Project risks}
\label{table:risks}
\begin{tabularx}{\textwidth}{|l|X|l|l|}
\hline
\rowcolor{Gray} \textbf{Risk ID} & \textbf{Description} & \textbf{Probability} & \textbf{Impact} \\
\hline
P1 & Short term absence & 3 & 1 \\
\hline
P2 & Long term absence & 2 & 1 \\
\hline
P3 & Someone drops out & 2 & 2 \\
\hline
T1 & Someone force-pushes to Gitlab & 2 & 1 \\
\hline
T2 & Processing turns out to be a very bad choice & 1 & 3 \\
\hline
C1 & Customer changes requirements & 3 & 2 \\
\hline
M1 & Divide between management and personnel & 1 & 3 \\
\hline
\end{tabularx}
\end{table}

\subsection{Personnel risks}

\subsubsection{Risk P1: short term absence of one person}

\begin{tabularx}{\textwidth}{rX}
\textbf{Root cause:} & A member will be absent for several days. \\
\textbf{Importance:} & Little importance, one of us can manage the project on their own anyway. \\
\textbf{Avoidance:} & It would still be nice to warn the project members so that we don’t rely on the concerned person to do any work. \\
\textbf{Response:} & Someone else takes responsibility for the person’s work. \\
\end{tabularx}

\subsubsection{Risk P2: long term absence of one person}

\begin{tabularx}{\textwidth}{rX}
\textbf{Root cause:} & A member will be absent for several weeks. \\
\textbf{Importance:} & Similarly to a short term absence, we can manage without one person for a prolonged period of time. \\
\textbf{Avoidance:} & A warning will do. \\
\textbf{Response:} & Someone else takes responsibility for the person’s work. \\
\end{tabularx}

\subsubsection{Risk P3: someone drops out}

\begin{tabularx}{\textwidth}{rX}
\textbf{Root cause:} & A member drops out of the course \\
\textbf{Importance:} & This is slightly more impactful then a prolonged absence. \\
\textbf{Response:} & We will have to reorganize the project around the three or fewer remaining members. \\
\end{tabularx}

\subsection{Technology risks}

\subsubsection{Risk T1: someone force-pushes to Gitlab}

\begin{tabularx}{\textwidth}{rX}
\textbf{Root cause:} & Lack of knowledge in the technology brings someone to erase all progress on Gitlab. \\
\textbf{Importance:} & Little importance, other members will have a copy of the project’s history. \\
\textbf{Response:} & Find out whoever has the most recent copy and push again. \\
\end{tabularx}

\subsubsection{Risk T2: Processing turns out to be a very bad choice}

\begin{tabularx}{\textwidth}{rX}
\textbf{Root cause:} & As we iterate over the project, we realise Processing will severely hinder our progress moving forward. \\
\textbf{Importance:} & Extemely unlikely given our constraints. \\
\end{tabularx}

\subsection{Customer risks}

\subsubsection{Risk C1: customer changes requirements}

\begin{tabularx}{\textwidth}{rX}
\textbf{Root cause:} & The customer changes their mind on a part of the project. \\
\textbf{Importance:} & Will depend on the size of the change. \\
\textbf{Response:} & Create or modify user stories, rework and refactor the concerned parts of the project. \\
\end{tabularx}

\subsection{Management risks}

\subsubsection{Risk M1: divide between management and personnel}

\begin{tabularx}{\textwidth}{rX}
\textbf{Root cause:} & The management and the personnel do not see eye-to-eye. \\
\textbf{Importance:} & Very unlikely considering the management and the personnel are one and the same on this project. \\
\end{tabularx}

